\chapter{Example of a study}
\label{sec:study}

\section{Preamble}
\label{sec:ex-preamble}

\begin{lstlisting}
prefix="."
patterns="./cocci"
projects="/home/lotfi/RICM4/stage/dev/herodotos/herodotos"
results="./results"
website="./website"
findcmd="spatch %f -sp_file %p -dir %d 2> %b.log > %o"
findchild = 8
flags="-timeout 60 -use_glimpse"
cpucore = 1
\end{lstlisting}
In this section, infomation necessary to indicate the work environment are indicated, like semantic patches,
results, and projects pathes , and the command to execute.
Other parameters can be givern, like the number of cores, to paralellize matching pattern step, instanciated in 
cpucore parameter, and the number of processes that coccinelle can run , given in findchild parameter.
Parameters in the command, indicated by a \% represent:
  \%f : Execution parameter, here the timeout is indicated.
  \%p : The semantic patch to apply.
  \%d : The directory where to study version code is located.
  \%o : The resulting report file.
  The three last parameters can be fixed in a Make file.

\section{Projects}
\label{sec:ex-projects}


\begin{lstlisting}
 
project Test {
 dir = testDemo	
 local_scm = "test"
 versions = "^v[0-2]$"       
}
 
\end{lstlisting}
In this section, project relative path is indicated, and different versions to study are described.
In this example, versions are described with a regular expression: versions to study are v0,v1 and v2 versions.
In this case, some information, like versions code size, and release date, are missing. These information are recovered
thanks to the local source code manager, which corresponds to a git repository. Source code can also be recovered if
the source code of a versions has not been recovered yet. If a local repository does not exist yet, it can be created
with a distant git repository, communicated with the "public\_scm" flag. In this case, a local git repository is 
created its name is fixed by the "local\_scm" flag value. After that, missing code is recovered, release date and code 
size are computed and saved. This step is consequently done once for each version.
There are other ways to describe versions, without using  regular expressions. Here are some examples:

\begin{lstlisting}
 
project Test {
 dir = testDemo	
 ("v0",07/23/2013,10)
 ("v1",07/23/2013,11)
 ("v2",07/23/2013,11)
}
\end{lstlisting}
This kind of declaration is necessary when there is no available source code manager, assuming that source code
is already present. In this case, the only information which can miss is code size.

\begin{lstlisting}
project Test2 {
	
	dir = testDemo
	local_scm = "test"
	versions = {
              ("v0",)
              ("v1",)
              ("v2",)
        }
}
\end{lstlisting}
This declaration is equivalent to the fist one, since versions names are listed without other information, instead of
being defined with a regular expression.

\section{Patterns}
\label{sec:ex-patterns}


\begin{lstlisting}
 
 pattern Error {
 file = "error.cocci"
 correl = strict
}


pattern Notes {
  file = "notes.cocci"
}
 
\end{lstlisting}

\section{Experiences}
\label{sec:ex-experiences}


\begin{lstlisting}

experience exp applying pattern Error,Notes
               with project Test, Test2		


\end{lstlisting}


